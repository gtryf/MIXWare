\font\booktitlefont=cmssdc10 at 46pt
\font\chapterfont=cmr10 at 28pt
\def\clos#1{{\rm clos}\left(#1\right)}
%
\pageno=-1
\font\auth=cmssdc10 scaled\magstep4
\font\elevenbf=cmbx10 scaled\magstephalf
\font\elevenit=cmti10 scaled\magstephalf
\font\elevenrm=cmr10 scaled\magstephalf
\titlepage
\line{\booktitlefont\hfill The MIXWare}
\smallskip
\line{\booktitlefont \hfill Report and Manual}
\vskip 1pc
\baselineskip 13pt \elevenbf
\halign to\hsize{#\hfil\tabskip 0pt plus 1fil&#\hfil\tabskip0pt\cr
\kern4.4mm\auth GEORGE \kern-1pt TRYFONAS&
 \tt george.tryfonas@gmail.com\cr}
\vfill
\eject

% blank page
\titlepage
\null\vfill
\eject

% the introduction
\titlepage
\def\rhead{Introduction}
\vbox to 8pc{
\rightline{\titlefont Introduction}\vss}
{\topskip 9pc
\vskip-\parskip
\tenpoint\noindent\hang\hangafter-2
\smash{\lower12pt\hbox to 0pt{\hskip-\hangindent\chapterfont T\hfill}}\hskip-16pt
{\sc HE} \MIX\ computer is an imaginary computer invented by Don 
Knuth in the 1960s in order to present his exposition on computer algorithms.
As Knuth puts it, the use of an imaginary computer and machine language helps
avoid distracting the reader with the technicalities of one particular computer
system, and the focus remains on truths that have always been---and will always
be---valid, independent of any kind of technological evolution or current trends.

There is no doubt about the truth of this statement. However, another kind of problem
presents itself now. The \MIX\ computer is, well, $\ldots$imaginary. A reader cannot 
experiment with it, or even have a go at solving the exercises by sitting in front 
of a real computer terminal and writing programs. Nor even can one be certain
that a particular solution to an exercise is correct, unless one checks the answer
to the exercise, and even then, there may be many different solutions to a 
problem, but the answers present only one. Unless of course the reader is 
inclined to simulate \MIX\ in his or her head, or on a piece of paper. And these 
may very well have been Knuth's intentions for his readers. After all, there is no
doubt that the reader that will get the most out of the books is the one who is
patient and dilligent enough to go through this kind of process. But it still is a 
daunting task, often too frustrating, and for many people in many ways, a 
distraction of the kind that Knuth wanted for his readers to avoid in the first place.

Apparently, these issues have been prevailing for quite some time, since many people
have developed simulators for Knuth's mythical machine over the years. The software
you now have on your hands is another such attempt. But why bother making yet another
simulator, you may ask? That is a good question. This present attempt in no way claims
to solve any problems that previous implementations do not. As a matter of fact, it
may have less bells and whistles than most. It might be less friendly, less complete,
or even slower in its calculations. But it delivers, correctly. And what has actually
been implemented, was done exactly by the book, literally speaking.

The question remains, then. Why another \MIX\ simulator? Here is the answer:
for the fun of it. And the learning experience. Writing such a piece of software,
however trivial it may seem, presents a number of great challenges. It is an
exercise in compiler design, input\slash output, data structures and more.
A computer simulator is a platform which, by definition, provides its clients
with {\it everything that the actual computer provides}, imaginary or not.
Furthermore, it must be done in the best possible way given the programming
tools.

The MIXWare are written in Microsoft Visual C\# v3.5, which is a pure object-oriented
language.

\medskip
\line{{\sl Athens, Greece}\hfil---G. T.}
\line{\sl June 2009\hfil}

}
\endchapter

'Tis the gift to be simple, 'tis the gift to be free.
\author{JOSEPH BRACKETT}, {\sl Simple Gifts} (1848)

\bigskip

Everything should be made as simple as possible,
but no simpler.
\author{ALBERT EINSTEIN}, {\sl On the Method of %
   Theoretical Physics} (1933)
\eject