\beginchapter Chapter 1. MIXSIM---\\The MIX Simulator

\pageno=1
\MIX, the \MIX\ simulator, incorporates all of the features described
in~{\it TAOCP, sec.~\hbox{1.3.1}}, with the exception of floating-point
arithmetic (which is not described in that section anyway, except by name).
No extensions have been implemented, specifically there is no support for
indirect addressing or for interrupts. These extensions are planned for a
future version.

Here's a quick overview of a \MIX\ computer's features:\smallskip
\item{$\bullet$} 4000 words of memory;
\item{$\bullet$} two word-sized registers called A and X;
\item{$\bullet$} six signed registers of size two bytes each, called I1 through I6;
\item{$\bullet$} one unsigned register of size two bytes, called J;
\item{$\bullet$} a comparison indicator, which can have the value GREATER, EQUAL
or LESS;
\item{$\bullet$} an overflow flag;
\item{$\bullet$} an instruction set of about 150 instructions;
\item{$\bullet$} 21 optional input\slash output devices: eight tape drives, 
eight disks, a card reader, a card punch, a line printer, a terminal and
paper tape, all of which are asynchronous.

\smallskip\noindent
Each \MIX\ word is comprised of five bytes and a sign. One byte is equal to
six bits, thus being able to have~64 distinct values.

\subsection RUNNING THE SIMULATOR.
The \MIX\ simulator can either run interactively or non-interactively. In
non-interactive mode, it simply loads a compiled \MIX\ program and 
runs it, producing whatever output the program is meant to produce, and
then terminates. To run \MIX\ non-interactively, type
\begindisplay
|MIX --deck [filename]|&or&|MIX --binary [filename]|\cr
\enddisplay
where \<filename> is optional and if not given the standard input is
assumed. The |--deck| and |--binary| switches instruct \MIX\ to interpret
the input as punched cards (|--deck|) or a binary memory dump (|--binary|).

In interactive mode, one may load and run programs, or execute them
step by step, or even set execution breakpoints and inspect the
machine's state---the values of the registers, the comparison indicator,
the program counter~{\sl etc}. To run the simulator interactively,
simply type |MIX| at the command prompt. The simulator welcomes
you:
\begintt
This is MIX v0, (c) 2009 George Tryfonas
An mplementation of the machine described by Don Knuth.
|vskip1em
Type '?' or 'help' at the prompt for instructions.
>
\endtt
The |>| character is the simulator's prompt for your next command.
You may quit the simulator by typing `|quit|' or `|bye|'.
Commands are case-insensitive and can be categorized as follows:
\smallskip
\item{$\bullet$} execution commands (take care of loading programs
to be executed and handle program execution);
\item{$\bullet$} inspection commands (for inspecting the contents
of simulated memory and registers);
\item{$\bullet$} state altering commands (that allow you to change
the contents of simulated memory and registers);
\item{$\bullet$} device management commands (for inspecting the state
of the devices and redirecting input\slash output).

\smallskip\noindent
The following sections provide a detailed overview of the
available commands.

\subsection EXECUTION COMMANDS.
There are several commands that control the simulator's program
loading and execution. Here is a list of them, and what they do.
\medskip
\item{|load <filename>|}Loads a binary memory dump
from the file \<filename> and sets up the program execution.
Firstly, it sets \MIX's memory contents to the contents of the
memory dump; and then it sets the contents of the program counter
to the location of the instruction where execution is to begin.
\item{}When the simulator loads the memory dump, control is returned to
you after a confirmation message. For example:
\begintt|hsize=31.5pc
> load primes.dump
Loaded 40 word(s). PC set to 3000
>
\endtt
at which point you may begin execution, set breakpoints, {\sl etc}.
\smallskip
\item{|loaddeck <filename>|}Instructs the simulator
to load the first card from the card deck specified in \<filename>
and begin execution at location $0$. This implies that the card
reader device (device~$16$) is redirected to \<filename>. Since
program execution begins immediately, you must set the desired
breakpoints, if any, before using `|loaddeck|'.
\item{}The reason for this is that `|loaddeck|' behaves in the way described
in~{\it TAOCP, sec.~\hbox{1.3.1}, ex.~$26$}: the first two cards in the
deck contain a loading routine, whose purpose is to load the rest of the
program into the correct memory locations and jump to the program
beginning. Therefore, no part of the program itself is loaded until this
loader routine has executed completely.\smallskip
\item{|go| or |run|}Begin execution at the location
pointed by the program counter. Execution does not stop unless it
reaches a breakpoint.\smallskip
\item{|step|}Execute the instruction pointed by the program counter.
Only one instruction is ever executed.
\smallskip
\item{|show time|}This command shows how much time has been spent on
program execution. Each \MIX\ instruction takes a specific amount of simulated
time to execute, so for instance the PRIMES program takes $190908u$, where
$u$ is an unspecified time unit.\smallskip
\item{|set breakpoint|}Set a breakpoint at a memory location. The
command `|set breakpoint <x>|' (or `|set bp <x>|' for short) sets
an execution breakpoint at memory location \<x>. \<x> can be a
number from~$0$ to~$3999$ or, if you loaded a memory dump, a symbol.
Each breakpoint that gets |set| is assigned a unique integer identifier.
\smallskip
\item{|show breakpoint|}Shows a list of all the breakpoints you have
set, along with their unique identifier:
\begintt|hsize=31.5pc
> set bp START
> set bp 3006
> show bp
0 @ 3000
1 @ 3006
>
\endtt
In this example, |START| is a symbol for the value $3000$. You may
also issue |show bp <n>| where \<n> is an integer, which will show
only the breakpoint whose unique identifier is equal to~\<n>.
\smallskip
\item{|clear breakpoint|}Clears a breakpoint that has previously been
set. The syntax is |clear breakpoint [n|\||all]|, where \<n> is an
integer. If \<n> is specified, then the breakpoint whose unique
identifier is equal to~\<n> is removed from the breakpoint list.
Otherwise, if~\<n> is not specified, or if `|all|' is specified
instead of~\<n>, all the breakpoints in the breakpoint list
are removed.

\subsection INSPECTION COMMANDS.
The simulator gives you the chance to inspect the machine's state using
a host of inspection commands, all of which begin with `|show|':\medskip
\item{|show symbols|}When a memory dump has been loaded (and only a
memory dump), you may take a look at the symbol table by issuing the
`|show symbols|' command. You may also issue a `|show symbol x|', where
\<x> is a symbol name, to inspect just the value of that particular
symbol. Example:
\begintt|hsize=31.5pc
> show symbols
L      = [+||00||00||00||07||52||] =  500 = '   G@'
OUTDEV = [+||00||00||00||00||18||] =   18 = '    Q'
PRIME  = [-||00||00||00||00||01||] =   -1 = '    A'
BUF0   = [+||00||00||00||31||16||] = 2000 = '   1O'
BUF1   = [+||00||00||00||31||41||] = 2025 = '   1,'
START  = [+||00||00||00||46||56||] = 3000 = '   *?'
||2-1||  = [+||00||00||00||46||59||] = 3003 = '   *?'
||4-1||  = [+||00||00||00||46||62||] = 3006 = '   *?'
||6-1||  = [+||00||00||00||47||00||] = 3008 = '   / '
||2-2||  = [+||00||00||00||47||08||] = 3016 = '   /H'
||2-3||  = [+||00||00||00||47||11||] = 3019 = '   /J'
||4-2||  = [+||00||00||00||47||12||] = 3020 = '   /K'
TITLE  = [+||00||00||00||31||11||] = 1995 = '   1J'
=1-L=  = [+||00||00||00||32||02||] = 2050 = '   2B'
=3=    = [+||00||00||00||32||03||] = 2051 = '   2C'
>
\endtt\smallskip
\item{|show memory|}You may inspect the contents of \MIX's memory by issuing
the `|show memory|' command (or `|show mem|' for short). This command, if
given without any arguments, shows the contents of all of \MIX's memory,
starting at location~$0$ and up to~$3999$. If, however, you specify a
range, you will get a portion of the memory contents:
\begintt|hsize=31.5pc
> show mem 3000 3010
MEMORY CONTENTS (3000 TO 3010)
|vskip1em
@3000: [+||00||00||00||18||35||] =       1187 = '   Q5'
@3001: [+||32||02||00||05||09||] =  537395529 = '2B EI'
@3002: [+||32||03||00||05||10||] =  537657674 = '2C E?'
@3003: [+||00||01||00||00||49||] =     262193 = ' A  $'
@3004: [+||07||51||01||05||26||] =  130814298 = 'G>AEW'
@3005: [+||47||08||00||01||41||] =  790626409 = '/H A,'
@3006: [+||00||02||00||00||50||] =     524338 = ' B  <'
@3007: [+||00||02||00||02||51||] =     524467 = ' B B>'
@3008: [+||00||00||00||02||48||] =        176 = '   B='
@3009: [+||00||00||02||02||55||] =       8375 = '  BB''
@3010: [-||00||01||03||05||04||] =    -274756 = ' ACED'
>
\endtt
Ranges are pairs \<$x$\]$y$>, for instance, 
$3000$\]$3010$ as in the above example. However $x$ or $y$ are
not limited to integers: symbols may be specified as well, as
in |show mem START 3010|, which produces exactly the same output
as above, given that |START| is a symbol whose value equals~$3000$.
\item{}You may also wish to see a disassembly of the memory contents
in the range specified. To do this, type `|with disassembly|'
(or `|with dasm|' for short) after a range---and only after
a range:
\begintt|hsize=31.5pc
> show mem START 3010 with disassembly
MEMORY CONTENTS (3000 TO 3010)
|vskip1em
@3000: [+||00||00||00||18||35||] =       1187 = '   Q5'       IOC 0(2:2)
@3001: [+||32||02||00||05||09||] =  537395529 = '2B EI'       LD1 2050
@3002: [+||32||03||00||05||10||] =  537657674 = '2C E?'       LD2 2051
@3003: [+||00||01||00||00||49||] =     262193 = ' A  $'       INC1 1
@3004: [+||07||51||01||05||26||] =  130814298 = 'G>AEW'       ST2 499,1
@3005: [+||47||08||00||01||41||] =  790626409 = '/H A,'       J1Z 3016
@3006: [+||00||02||00||00||50||] =     524338 = ' B  <'       INC2 2
@3007: [+||00||02||00||02||51||] =     524467 = ' B B>'       ENT3 2
@3008: [+||00||00||00||02||48||] =        176 = '   B='       ENTA 0
@3009: [+||00||00||02||02||55||] =       8375 = '  BB''       ENTX 0,2
@3010: [-||00||01||03||05||04||] =    -274756 = ' ACED'       DIV -1,3
>
\endtt
\item{|show state|}Shows \MIX's internal state. It produces
a list with the contents of all registers, $r$A, $r$X, $r$J, $r$I\<$1$--$5$>,
the overflow flag, the comparison indicator and the program counter:
{\begintt |hsize=31.5pc
> show state
 A: [+||00||00||00||00||00||] = 0 = '     '
 X: [+||00||00||00||00||00||] = 0 = '     '
 J: [+||00||00||00||00||00||] = 0 = '     '
I1: [+||00||00||00||00||00||] = 0 = '     '
I2: [+||00||00||00||00||00||] = 0 = '     '
I3: [+||00||00||00||00||00||] = 0 = '     '
I4: [+||00||00||00||00||00||] = 0 = '     '
I5: [+||00||00||00||00||00||] = 0 = '     '
I6: [+||00||00||00||00||00||] = 0 = '     '
Overflow: False
CI: EQUAL
PC: 3000
>
\endtt}%
You may also issue the command `|show <reg>|' where \<reg> can be any one
of the above registers or flags to see the contents of that specific item.
\item{|verbose|}Toggles verbosity on and off. When verbose is on, after
each `|step|' command, and after program termination following a `|run|'
command, a `|show state|' is executed automatically. You may check whether
verbose is on or off by issuing the `|show verbose|' command.

\subsection STATE ALTERING COMMANDS.
There exist various commands which alter the simulator's state between
execution. All of these commands begin with `|set|'. Here is a list of these:
\medskip
\item{|set memory|}Alters the contents of a single memory cell. The syntax is
|set memory <loc> <data>|, where \<loc> is a memory location and \<data> is
an integer value. As is the case with |show memory|, the \<loc> parameter
may be a symbol.\smallskip
\item{|set <reg> <value>|}Alters the contents of one of \MIX's registers.
The \<value> parameter is an integer while the \<reg> parameter can be one
of the nine \MIX\ registers or the program counter, in the form |rA|, |rX|, 
|rJ|, |rI1| to~|rI6| and |PC|.

\subsection DEVICE MANAGEMENT COMMANDS.
\MIX\ has an array of input\slash output device units that programs may use.
The most common one is the line printer, however \MIX\ is also supplied with
tape drives, disks, a card reader, a card punch and a paper tape. All of
these devices have been implemented and you may inpspect or alter their
status using the following commands:\medskip
\item{|show devices|}Shows a list of \MIX's devices. For example:
\begintt|hsize=31.5pc
> show devices
UNIT #0: NOT INSTALLED
UNIT #1: NOT INSTALLED
UNIT #2: NOT INSTALLED
UNIT #3: NOT INSTALLED
UNIT #4: NOT INSTALLED
UNIT #5: NOT INSTALLED
UNIT #6: NOT INSTALLED
UNIT #7: NOT INSTALLED
UNIT #8: NOT INSTALLED
UNIT #9: NOT INSTALLED
UNIT #10: NOT INSTALLED
UNIT #11: NOT INSTALLED
UNIT #12: NOT INSTALLED
UNIT #13: NOT INSTALLED
UNIT #14: NOT INSTALLED
UNIT #15: NOT INSTALLED
UNIT #16: NAME: CARD_READER, BLOCK SIZE: 16, BACKING STORE: CONSOLE
UNIT #17: NOT INSTALLED
UNIT #18: NAME: LINE_PRINTER, BLOCK SIZE: 24, BACKING STORE: CONSOLE
UNIT #19: NAME: TERMINAL, BLOCK SIZE: 14, BACKING STORE: MEMORY
UNIT #20: NOT INSTALLED
>
\endtt
which displays various kinds of information for each device. In this
specific example, there are no installed tape or disk units, card punch
or paper tape. However the card reader, the line printer and the
terminal are present. The first two are connected to the console---so
if a program sends output to the line printer it is displayed on the
screen, and if it request input from the card reader it is read from
the keyboard. You may also issue the command |show device <d>| where
\<d> is an integer from~$0$ to~$20$, indicating a unit number, to
get information on that specific device unit.\smallskip
\item{|redirect|}Redirects a device to a specified file. The syntax is:
\begindisplay\hsize=31.5pc
|redirect <d> <filename>|&or&|redirect <d> console|
\enddisplay
where \<d> is an integer from~$0$
to~$20$ representing the device unit. If a filename is specified then
input\slash output is done using that file. Otherwise if you say
|console| then input\slash output is done using the standard input\slash
output. You cannot redirect a device to a file called |console|.
\item{|set device|}Alters the synchronicity of a device. The syntax is
\begindisplay\hsize31.5pc
|set device <d> sync|&or&|set device <d> async|
\enddisplay
where \<d> is the device unit number. |sync| and |async| specify the
mode in which device \<d> will operate. By default each device is
asynchronous, which means that as soon as an input\slash output
operation is executed by a running program, control is returned to
that program and the input\slash output operation continues to
execute in the background. Note that this is not in the formal
specification of \MIX.
\endchapter

Japhy, do you think God made the world to amuse himself
because he was bored? Because if so he would have to be mean.
\author{JACK KEROUAC}, {\sl The Dharma Bums} (1958)

\eject