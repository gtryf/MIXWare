\beginchapter Chapter 3. Simulator Implementation

We shall inspect, in this and the next chapter, the various details behind the
two principal programs, |MIXASM| and |MIX|. These chapters are concerned a bit 
more with the hairy implementation details and a good working knowledge of 
object oriented programming is assumed. The actual implementation language is
C\#, but we shall avoid getting into actual programming details. However, will
will talk about classes, structures, methods and inheritance, and the reader is
assumed to know what each of these constructs means.

We start our discussion with the simulator, which is probably the simplest
of the two programs, therefore easier to grasp.
\medskip
The simulator closely follows the specification given in {\it TAOCP sec. 1.3.1.}
Everything described in that section has been implemented in its entirety.
Perhaps the most challenging task of the implementation process is the fact
that \MIX\ is an archaic design. Many of its components do not exist today.
Certainly there are no card readers, card punches or tape drives. Printers
have stopped being line-oriented. Byte size has been standardized and it
certainly isn't six bits. Computers generally make no distinction between
$+0$ and $-0$.

Object oriented design helps hide such minor setbacks. We recognize the
abstractions behind the computer specification and create classes to encapsulate
them. For example, we know that a word represents a number, of any radix,
whose minimum and maximum limits are known, and on which we may perform simple
arithmetic. We are not concerned with how many bits represent one word. This is a
minor implementation detail which is kept hidden within the encapsulating class.

The same goes with the other important parts of the \MIX\ design.
Breaking a \MIX\ computer to its abstract components, we are left with the
following important classes:
$$
\vbox{\tabskip 3em plus1em minus.5em
\halign to \hsize{&#\hfil\cr
\omit\hfil\MIX\ Abstraction\hfil&\omit\hfil Encapsulating\hfil&\omit\hfil Remarks\hfil\cr
&\omit\hfil Class\hfil\cr
\noalign{\smallskip\hrule\medskip}
Word&|MIXWord|&A word of memory: five bytes and a sign\cr
Instruction&|MIXInstruction|&An abstract \MIX\ instruction, like |LDA|.\cr
&&Consists of an |Execute()| method,\cr
&&which when overridden performs the\cr
&&actual instruction.\cr
Device&|MIXDevice|&An abstract \MIX\ device.\cr
&&Descendants are the concrete \MIX\ devices,\cr
&&such as disk drives, the line printer, {\it etc.}\cr
\MIX&|MIXMachine|&The actual \MIX\ computer.\cr
&&Contains properties such as |Memory|, a set\cr
&&of register properties, the device array {\it etc.}\cr
}}
$$
\smallskip\centerline{\ninerm{\ninebf Table 3.1} \MIX\ Description}\smallskip
Figure~3.1 shows the overall structure of the \MIX\ simulator design without many
details. The diagram drives home the fact that the |MIXMachine| class is the one
class that brings together the various parts that make up \MIX.

\epsfxsize=.9\hsize
\centerline{\epsfbox{eps/machine.eps}}
\smallskip\centerline{\ninerm{\ninebf Figure 3.1} \MIX\ Simulator class diagram}
\smallskip\noindent
We now shall discuss the various parts of the \MIX\ architecture as evidenced
in the class diagram in a bit more detail.
\subsection WORDS.
\MIX\ words are used throughout the specification of the \MIX\ machine:
\MIX's memory is an array of \MIX\ words; the \MIX\ registers hold values that
fit into \MIX\ words (in fact, the index registers, rI1--rI6, as well as rJ,
are shorter, but regarding them as such is a good simplification); each \MIX\
instruction is encoded as a word; device I/O is done by sending and receiving
words to and from the devices. It is a ubiquitous abstraction.

It follows that such an ever-present abstraction will make very frequent
appearances in any kind of implementation. To this end, we go at great length
to define the |MIXWord| class in such a way that it may be used as naturally as 
the standard integral types of the implementation language. The |MIXWord| class 
can be constructed from the language's existing integral values (|int| or |byte|) 
and can be cast back to them. In the case of |int|, an explicit cast is needed 
because the range of a 32-bit integer is greater than that of the 30-bit \MIX\
word. In addition, we overload the standard C\# arithmetic and relational operators
in such a way that we may write
\begintt
if (w1 > w2)
    w1 += 100;
\endtt
Note that because
of our type-coercion operators we are allowed to add a |byte| (such as 100)
to a |MIXWord|.

An important characteristic of a \MIX\ word is that we can examine the value of,
and assign to, byte slices of it. For example, we can assign the value `1000'
to bytes 2--3 of a word, leaving the rest of the word the same. Or, we can
read the value of bytes 2--4, ignoring the other bytes. These byte slices
are called {\it fields}.

Because fields are contiguous, it is simple to encapsulate them in the |MIXWord|
class by overloading the |this[]| property. We provide two overloads:\smallskip
\item{1.}|public byte this[byte b]| assigns to or reads the value of a single
byte of the |MIXWord|; and
\item{2.}|public int this[byte l, byte r]| assigns to or reads the value of
a field of the |MIXWord|, defined by the range $l$--$r$, where $l$ is the most
significant byte of the field and $r$ is the least significant byte.
\smallskip\noindent
therefore providing the means to write code such as:
\begintt
MIXWord w = new MIXWord(5000);
w[2,3] = 1000;
int a = w[1,4];
\endtt
In this way we make the implementation of the \MIX\ instruction set significantly
easier, because the effective address of most instructions takes a field
specification into account.

Finally, we override the |ToString()| method to return a convenient textual
representation of the word, useful in program listings and in memory dumps.

\subsection INSTRUCTIONS.
A \MIX\ instruction is encoded in a word according to the conventions specified
in {\it TAOCP sec.~1.3.1}. The \MIX\ central processing unit decodes the word
and, according to the C- and F-fields of the instruction word (see also the
section on list files in the previous chapter), performs a specified
operation using |CONTENTS(M)|, the latter defined as
$$ {\rm CONTENTS}(M)\equiv A(F)+{\rm rI}(I) $$
where $A$, $F$ and $I$ are the A-, F- and I- fields of the instruction word
respectively.

To encapsulate this abstraction in the |MIXInstruction| class there are two
distinct paths we may follow. We can either: (1)~create an abstract superclass
and inherit it 157 times, once for each instruction; or (2)~create a concrete
class and instantiate it with a different method delegate 157 times, again once
for each instruction. The first method encapsulates the abstraction at the class
level while the second one at the instance level.

There are no significant advantages of one method over the other, apart from
the fact that the first one leads to an explosion of subclasses, each one of
which must be named differently, while the second method can be implemented
using anonymous delegates ($\lambda$-expressions). We chose the second method
for this simple reason.

\subsection DEVICES.
As required by the specification, the \MIX\ computer features asynchronous
I/O. All devices communicate with the computer by sending and receiving
blocks of data, either as full \MIX\ words (in the case of disks and magnetic
tape) or in character mode (for the other devices). In character mode the
sign of the \MIX\ word is ignored and bytes 1--4 are interpreted as a characters
from the \MIX\ character set. Block size varies from device type to device type.

This abstraction is encapsulated in the |MIXDevice| abstract class and its
subclasses, which are illustrated in Figure~3.2 below:

\epsfxsize=.9\hsize
\centerline{\epsfbox{eps/devices.eps}}
\smallskip\centerline{\ninerm{\ninebf Figure 3.2} \MIX\ Devices class diagram}
\smallskip\noindent
The |MIXDevice| abstract class contains the core interface to a \MIX\ device.
Its important methods are |In()|, |Out()| and |IOC()|. These methods
simply call the |InProc()|, |OutProc()| and |IOCProc()| methods respectively
in a new thread of execution---this provides the asynchronous device operation
required by the specification.

Concrete children of the |MIXDevice| class override these last three methods
to provide the functionality of the implemented device. Some devices are meant
only for input, others only for output, in which case the |InProc()| or the
|OutProc()| methods respectively have null implementations.

The |Ready| property is the device's ``status indicator''. It is cleared
by the |InProc()|, |OutProc()| and |IOCProc()| methods whenever they begin
execution, and reset when they end. It is important that the body of these
methods is written inside a critical region to provide the appropriate
synchronization. C\#'s |lock| primitive is used for this purpose.

Devices are actually implemented using streams as backing store. Upon instantiation
of a |MIXDevice| subclass, we specify a |Stream| instance for such use,
but it's possible to redirect a device by simply changing the underlying
stream, to either a |MemoryStream|, the standard input/output streams or
a |FileStream|---or |null| if the device is unbacked, making the device behave
as if it did not exist at all. The |Redirect()| method of the |MIXDevice| class
is provided exactly for this purpose.

\subsection THE MACHINE.
Everything is put together in the |MIXMachine| class. Details such as the
\MIX\ character map, the register set (a set of |MIXWord| properties), the
overflow/comparison indicators, the memory (an array of |MIXWord|s), the
device array and the instruction set are all implemented as properties
in this class.

A |MIXMachine| instance can load a binary memory dump or a card deck as they
are output from the assembler. There exist methods to execute one instruction
or to execute instructions sequentially until an |HLT| instruction is reached.
Execution is governed by a special property called |PC| (for |P|rogram |C|ounter)
which holds the memory location of the next instruction to be executed.
\endchapter

One Ring to rule them all, One Ring to find them,
One Ring to bring them all and in the darkness bind them
In the Land of Mordor where the Shadows lie.
\author{J. R. R. TOLKIEN}, {\sl The Fellowship of the Ring} (1954)
\eject