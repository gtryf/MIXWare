\beginchapter Chapter 2. MIXASM---\\The MIX Assembler

The assembler implements all the features of the \MIXAL\ language as
detailed in~{\sl TAOCP, sec.~1.3.2}. See that book section for
the definitive description of the language. The assembler
follows the book faithfully, including the format of the
|ALF| pseudo-operation. Appendix~A has a detailed treatment of
the exact semantics of this particular assembler implementation.

\subsection RUNNING THE ASSEMBLER.
The assembler is invoked using the following syntax:
\begintt
MIXASM [arguments] [input-filename]
\endtt
where each \<argument> is of the form:
\begindisplay
|--long-arg:value|&or&|-short-arg:value|\cr
\enddisplay
where for each |--long-arg| there exist one or more |-short-arg|s
that perform the equivalent function but are, well, shorter. The
arguments' |value| part is, depending on the argument, either
required, optional, or not applicable.
Here's a list of the possible arguments and a description of
what they do. The applicability of the |value| part is also
discussed here.
\def\drule{\noalign{\smallskip}}
$$
\tabskip 1em plus .5em minus .5em
\halign to \hsize{#\hfil&#\hfil&\hfil#\cr
{\sl Long Form}&{\sl Short Form}&{\sl Description}\cr
\noalign{\vskip3pt\hrule\smallskip}
|--output:filename|&|-o:filename|&redirects output to |filename|.\cr
\drule
|--symtab[:filename]|&|-s[:filename]|&produces the symbol table. If |filename|\cr
&&is specified, the symbol table is written to that\cr
&&file, otherwise it is sent to standard output.\cr
\drule
|--list-file:filename|&|-lf:filename|&produces a listing and writes it\cr
&&to |filename| (see below).\cr
\drule
|--format:...|&|-f:...|&defines the output format. Can be either\cr
&&|Card| or |Binary| (default is |Binary|, see below).\cr
\drule
|--append-deck:filename|&|-a:filename|&When the output format is |Card|, the\cr
&&|filename| specified gets appended to\cr
&&the output card deck.\cr
\drule
|--pretty-print|&|-pp|&pretty prints the symbol table and\slash or\cr
&&the list file (if requested) for use with \TeX.\cr
\drule
|--version|&|-v|&display version information and exit.\cr
\drule
|--help|&|-h|\cr
&|-?|&display a short help message and exit.\cr
}
$$
If |input-filename| is not specified then input is done from the standard input.
If the |--output| parameter is omitted then the compiler emits output to the
standard output. The arguments are {\sl not\/} case-sensitive.
Here's a bunch of examples:
$$
\tabskip 1em plus .5em minus .5em
\halign to \hsize{#\hfil&\hfil#\cr
{\sl Command*}&{\sl Comments}\cr
\noalign{\vskip 3pt\hrule\medskip}
|MIXASM -f:card primes.mixal|&compiles the file |primes.mixal| to\cr
&a card deck and sends the output to the display\cr
\drule
|MIXASM -f:card primes.mixal|\cr
| -lf:primes.tex -pp|&as above, but also pretty-prints the program listing\cr
&for use with \TeX, and saves it to the file |primes.tex|.\cr
\drule
|MIXASM --output:primes.dump|\cr
|  < primes.mixal|&compiles the file |primes.mixal| and\cr
&writes the (binary) output to file |primes.dump|\cr
\drule
|MIXASM -f:card primes.mixal|\cr
|  |\|| MIX --deck|&compiles the file |primes.mixal| as a card deck\cr
&and starts non-interactive simulation\cr
\noalign{\smallskip\hrule\smallskip}
\multispan2\eightrm*These commands are, of course, all one-liners.\hfil\cr
}
$$
The last couple of examples are worth particular mention. They bring to
the light the joy of pipelining, which so many UNIX users have come to
worship. Here's why the third example works: |MIXASM|, unless told otherwise,
expects that the file to be assembled comes from the standard input. But
here we redirect the standard input to the file |primes.mixal|. In the
fourth example, we redirect |MIXASM|'s output to the input of |MIX|,
the simulator. Remember that the |--deck| argument informs the simulator
to interpret the input as a card deck.

The following sections provide a description of the rest of |MIXASM|'s
features (well, except for the dead obvious ones).

\subsection OUTPUT FORMATS.
|MIXASM| can produce two kinds of output, namely a binary memory dump
or a card deck. Memory dumps are not human-readable and are meant only to be
loaded and executed by |MIXSIM|, the simulator ({\sl cf.}~Chapter~1
and~Appendix~A for techincal details). Card
deck output, however, is in ASCII format and can be read by humans
too---well, at least those that make sense of it.

A card deck is a text file containing 80-column lines. Each line 
represents a punched card and is read by \MIX's card reader. Each 
\MIX\ program that has been compiled to a card deck adheres
to the specification given in~{\it TAOCP, sec.~\hbox{1.3.1}, ex.~$26$}:
The first two cards contain a loading routine which serves to load
the rest of the program into the correct locations in memory and
start execution. Note that programs that are compiled to cards
must start at a memory location greater than~$100$, a restriction
that does not apply when a program has been compiled to binary.
This is due to the fact that these locations are occupied by the
loading routine itself.

When a \MIXAL\ program is compiled to a card deck, one may specify
the |--append-deck| argument, which in effect appends extra cards to
the output deck. For example, typing
\begintt
MIXASM perms.mixal -o:perms.deck -f:binary -a:perms-input.deck
\endtt
will produce the card deck |primes.deck| which will contain the
compiled program plus the cards found in the |primes-input.deck|
card deck (all these files can be found in the {\it Examples}
directory of the MIXWare distribution). This latter deck contains
sample input data for the permutations program. 

There is actually a (good) reason behind this. If you use the
simulator's |loaddeck| command to load the compiled program,
the loader routine, program and all, will execute immediately,
giving you no chance to set up the card reader with any input
data (if there is any, of course). Therefore, the |--append-deck|
argument gives you a chance to set things up beforehand.

\subsection SYMBOL TABLES.
The argument |--symtab| instructs |MIXASM| to produce a symbol table.
If the |filename| parameter is given to the argument, then the symbol
table is written to that file, otherwise it is sent to the standard
output.
The symbols in \MIX's assembly language are divided into three
categories and so is the symbol table: (1)~the main symbols, which
are |EQU| constants and instruction labels; (2)~the local symbols,
i.e., labels like~|1H| and |9H| in the assembly source, and;
(3)~literals, i.e., the address parts of instructions with the
format |=23//1=|.

Each segment of the symbol table is divided into three columns.
The first column is the symbol name. The second one is the symbol's
value as it gets stored in a \MIX\ word. The third column is the
symbol's decimal value. Main symbols are copied verbatim as
described. Local symbols and literals, however, receive special
treatment.

Local symbols get renamed according to the following rule: when
a label of the form $n$|H| is encountered, where $n$ is a digit,
then that label is renamed to 
\|\kern-0.2112em$n$|-|$x_n$\kern-0.2112em\|, where $x_n$ is
initially equal to~$1$. If the assembler encounters the label
$n$|H| again, then it increases~$x_n$ by~$1$ and repeats the
renaming process. For example: when the assembler first encounters
a label |2H|, it sets $x_2$ equal to~$1$ and renames the label
to \||2-1|\|. If the assembler encounters the label again further
down the source code, it sets $x_2$ to~$2$ and renames the label
to \||2-2|\|.

Literals do not appear as symbols in the \MIXAL\ source. They appear as
the address part of \MIXAL\ instructions, for example in the 
instruction~|LD2 =3=|, the address~|=3=| is a literal constant. This
has the effect of creating a constant with the value $3$ and creating
an internal symbol whose value is equal to the address that constant
is stored. |MIXASM| names that symbol as the literal---in the given
example that symbol will be called~|=3=|. Note that this implies that
only one constant is created, regardless of how many times the
literal~|=3=| appears in the source code. There is, however, one
exception to this rule, and that is when the literal contains the
location counter, {\sl e.g.} |=*+3=|. See Appendix~A for more details.

\subsection LIST FILES.
|MIXASM| can produce list files similar to the program listings
in~{\sl The Art of Computer Programming}: for each line of source,
the listing contains the location in memory where that line is assembled, 
the assembled word, the line number and the line itself.
If the instruction
is a \MIXAL\ pseudo-instruction, the first two parts are omitted.
The program listing is followed by the symbol table as described
in the previous section. A typical \MIX\ instruction is stored in 
a \MIX\ word as follows:
$$
\global\usehruletrue\mixinst$\pm$.AA.I.F.C.
$$ where |AA| is the address, ranging from $0$ to~$3999$, |I| is
the index, |F| is the field and |C| is the instruction's opcode. For
more information on how these fields are used see~{\it TAOCP, sec.~\hbox{1.3.1}}.

The list file that |MIXASM| produces is stored in a text file that's
specified as an argument to the |--list-file| parameter. If the 
|--pretty-print| option is given to the assembler, the list file is
written in \TeX{} format instead. You may refer to Appendix~A for a full
listing of a short example program, to examine what such listings look like when
they are typeset by \TeX.
\endchapter

When someone says: `I want a programming language in which
I need only say what I wish done', give him a lollipop.
\author{ALAN PERLIS}, {\sl Software Metrics} (1981)

\bigskip

Trying to outsmart a compiler defeats much of the purpose of using one.
\author{KERNIGHAN \& PLAUGER}, {\sl The Elements of Programming Style} (1978)

\bigskip

If you can't do it in Fortran, do it in assembly language. 
If you can't do it in assembly language, it isn't worth doing.
\author{ED POST}, {\sl Real Programmers Don't Use Pascal} (1983)
\eject